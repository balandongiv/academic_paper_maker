\documentclass{article}
\usepackage{amsmath}
\usepackage{graphicx}
\usepackage{hyperref}

\title{Your Title Here}
\author{Your Name}
\date{\today}

\begin{document}

\maketitle

\section{Introduction}
Electroencephalography (EEG) is a neurophysiological monitoring method to record electrical activity of the brain. Originating from the pioneering work of Hans Berger in the 1920s, EEG has become a cornerstone in neuroscience and clinical neurophysiology.  The non-invasive nature of EEG, coupled with its ability to directly reflect neuronal dynamics, makes it a powerful tool for studying brain states and cognitive processes.  In modern society, driver fatigue is a significant contributor to traffic accidents, posing a severe threat to public safety.  Recognizing fatigue's detrimental effects on vigilance, reaction time, and decision-making, researchers began exploring objective and reliable methods for its detection.  Among various physiological measures, EEG emerged as a promising candidate for real-time fatigue monitoring in driving scenarios. The rationale is that fatigue induces detectable changes in brainwave patterns, particularly in frequency bands associated with drowsiness and reduced alertness.  Early studies in the late 20th and early 21st centuries laid the groundwork for utilizing EEG to identify fatigue-related neural signatures, paving the way for the development of EEG-based fatigue driving detection systems.

\end{document}
